\documentclass[a4paper,10pt]{article}
\usepackage[utf8]{inputenc}
\usepackage[margin=1in]{geometry}
\usepackage{graphicx}
\usepackage{listings}
\usepackage{hyperref}

\begin{document}

\begin{titlepage}
\begin{center}
\vspace*{1cm}

\huge{\textbf{Keystroke Dynamics}}

\vspace{0.5cm}
CS725 Project

\vspace{3.5cm}
Indian Institute of Technology, Bombay\\
Department of Computer Science and Engineering
\vspace{3.5cm}

\Large{Devashish Singh (163059001) \\ Prateek Patidar (163059006) \\ Shubham Singh (163059008) \\ Hareesh Kumar (16305R013)}

\vfill

\vspace{0.8cm}

\includegraphics[scale=0.35]{IITB.png}
\end{center}
\end{titlepage}
 
\section{Project Description}
Keystroke dynamics is the study of whether people can be distinguished by their typing rhythms, much like handwriting is used to identify the author of a written text. Possible applications include acting as an electronic fingerprint, or in an access-control mechanism. A digital fingerprint would tie a person to a computer-based crime in the same manner that a physical fingerprint ties a person to the scene of a physical crime. Access control could incorporate keystroke dynamics both by requiring a legitimate user to type a password with the correct rhythm, and by continually authenticating that user while they type on the keyboard. 

\section{Tentative Approach}
We'll proceed using the following workflow:
\begin{itemize}

 \item(training): Retrieve the first 200 passwords typed by the genuine user from the password-timing table. Use the anomaly detector's training function\cite{doi:10.1109/DSN.2009.5270346} and other functions with these password-typing times to build a detection model for the user's typing.

 \item(cross validation ): Retrieve the last 200 passwords typed by the genuine user from the password-timing table. Use the anomaly detector's\cite{doi:10.1109/DSN.2009.5270346}  scoring function and the detection model (from Step 1) to generate anomaly scores for these password-typing times. Record these anomaly scores as user scores.

Repeat the above four steps, designating each of the subjects as the genuine user in turn, and calculating the equal-error rate for the genuine user. Calculate the mean of all 51 subjects' equal-error rates as a measure of the detector's performance, and calculate the standard deviation as a measure of its variance across subjects. 

\end{itemize}




\section{Papers}
\begin{itemize}
 \item Comparing Anomaly Detectors for Keystroke Dynamics\cite{doi:10.1109/DSN.2009.5270346}
 \item ROCR: visualizing classifier performance in R \cite{doi:10.1093/bioinformatics/bti623}
\end{itemize}

\section{Datasets}
We will be using dataset from CMU. The dataset consists The data consist of keystroke-timing information from 51 subjects (typists), each typing a password (.tie5Roanl) a total of 400 times in 8 sessions. The dataset consists of digraph data for keystrokes.


\section{Work Done till Now}
We did some feature engineering on the dataset and used some classification methods available in scikit-learn library on the data like Logistic Regression, Support Vector Machines, Random Forests, K NN Classification and Gaussian Naive Bayes.


\bibliographystyle{unsrt}
\bibliography{References}


\end{document}
